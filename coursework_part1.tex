% #######################################
% ########### FILL THESE IN #############
% #######################################
\def\mytitle{Coursework Report Part 1}
\def\mykeywords{Deadalus Delicatessen, Monster Bakery, SET08101, Bruce McClelland}
\def\myauthor{Bruce McClelland}
\def\contact{40209907@live.napier.ac.uk}
\def\mymodule{Web Technology (SET08101)}
% #######################################
% #### YOU DON'T NEED TO TOUCH BELOW ####
% #######################################
\documentclass[10pt, a4paper]{article}
\usepackage[a4paper,outer=1.5cm,inner=1.5cm,top=1.75cm,bottom=1.5cm]{geometry}
\twocolumn
\usepackage{graphicx}
\graphicspath{{./images/}}
%colour our links, remove weird boxes
\usepackage[colorlinks,linkcolor={black},citecolor={blue!80!black},urlcolor={blue!80!black}]{hyperref}
%Stop indentation on new paragraphs
\usepackage[parfill]{parskip}
%% Arial-like font
\usepackage{lmodern}
\renewcommand*\familydefault{\sfdefault}
%Napier logo top right
\usepackage{watermark}
%Lorem Ipusm dolor please don't leave any in you final report ;)
\usepackage{lipsum}
\usepackage{xcolor}
\usepackage{listings}
%give us the Capital H that we all know and love
\usepackage{float}
%tone down the line spacing after section titles
\usepackage{titlesec}
%Cool maths printing
\usepackage{amsmath}
%PseudoCode
\usepackage{algorithm2e}

\titlespacing{\subsection}{0pt}{\parskip}{-3pt}
\titlespacing{\subsubsection}{0pt}{\parskip}{-\parskip}
\titlespacing{\paragraph}{0pt}{\parskip}{\parskip}
\newcommand{\figuremacro}[5]{
    \begin{figure}[#1]
        \centering
        \includegraphics[width=#5\columnwidth]{#2}
        \caption[#3]{\textbf{#3}#4}
        \label{fig:#2}
    \end{figure}
}

\lstset{
	escapeinside={/*@}{@*/}, language=C++,
	basicstyle=\fontsize{8.5}{12}\selectfont,
	numbers=left,numbersep=2pt,xleftmargin=2pt,frame=tb,
    columns=fullflexible,showstringspaces=false,tabsize=4,
    keepspaces=true,showtabs=false,showspaces=false,
    backgroundcolor=\color{white}, morekeywords={inline,public,
    class,private,protected,struct},captionpos=t,lineskip=-0.4em,
	aboveskip=10pt, extendedchars=true, breaklines=true,
	prebreak = \raisebox{0ex}[0ex][0ex]{\ensuremath{\hookleftarrow}},
	keywordstyle=\color[rgb]{0,0,1},
	commentstyle=\color[rgb]{0.133,0.545,0.133},
	stringstyle=\color[rgb]{0.627,0.126,0.941}
}

\thiswatermark{\centering \put(336.5,-38.0){\includegraphics[scale=0.8]{logo}} }
\title{\mytitle}
\author{\myauthor\hspace{1em}\\\contact\\Edinburgh Napier University\hspace{0.5em}-\hspace{0.5em}\mymodule}
\date{}
\hypersetup{pdfauthor=\myauthor,pdftitle=\mytitle,pdfkeywords=\mykeywords}
\sloppy
% #######################################
% ########### START FROM HERE ###########
% #######################################
\begin{document}
    \maketitle
    \begin{abstract}
        %Replace the lipsum command with actual text 
        \lipsum[2]
    \end{abstract}
    
    \textbf{Keywords -- }{\mykeywords}
    
    \section{Game Descriptor}
	My game is a free-to-play browser based adventure-RPG where the player controls a character. They select their gender,
	name, shirt colour and starting item which will be with them until used. There will be a myriad of game changing choices.These options have an impact on future dialogue as well as game length. The game can also be reset easily if a mistake was made or just for a more wholesome experience if the player wishes to experience all it has to offer. \\
    The main aim of the game is to revert monsters back to their original human form with the legendary power of perfect baking. Ingredients can be found within the kingdom or bakery if you so wish to pick them up and there will be multiple combinations available, different coloured icings, sponges etc. Some of these options are considered bad and can cause completely different negative outcomes. A few examples of this are as follows; the monsters lunge at you, go on a rampage or even grow in size. So make sure you're polite, bake and gift them the best cake you've ever made to appease those hungry monsters.
    
	\subsection{Adventure Based Game}
	Role Playing Game (RPG) is a genre that creates incredible depth and atmosphere whilst you take control of a character(s)
	within a fictional world. There is a lot of hybrid-genres available within the mothership genre, so it is a very
	challenging genere to describe.\\
	Some sub-genres include; 
	\begin{itemize}
		\item Strategy-RPG
		\item Adventure-RPG
		\item Online-RPG
		\item Action-RPG
	\end{itemize}
	For the purpose of this report, we will talk in greater detail about text-based adventure RPGs.
	Text-based adventure RPG games have been around for a very long time. It is largely story driven
	with events unfolding when players make selections from the multiple on-screen options.
	
	\subsection{Story Draft}
	\paragraph{Daedalus Delicatessen} is a small bakery right in the center of the bustling city Cornelia. One of the only
	human populated nations in planet Terra. Monsteria is a condition that has plagued Terra since the dawn of time where
	humans turn into blood thirsty beasts, some of which call Cornelia home. Of course this caused mass panic and with skin
	several times stronger than tungsten, the royal guards are not remotely effective. \\ 
	\hspace{5mm} Child of Queen Lavender, you are a world renowned baker that makes the most delicious cakes. With the
	ancestral power to slay the stomachs of even the fiercest creatures, you are the worlds only hope to restore humankind
	to it's former glory. To do this, you need to create the most irresistable cupcakes and become Cornelias only hope!
	
	\section{Project Methodologies}
	\subsection{Javascript Game Idea}
	Javascript, HTML and CSS are all very new to me. Especially Javascript, to remedy this I decided to research how to implement all of the above in a more game type environment. As	luck would have it, I found the perfect YouTube video created by Web Dev Simplified	\cite{JavascriptGame}. This has everything I need to make a very basic game including	variables to store collected items, different ways to end the adventure (death, completion) and from the ending you can select an option to restart the game. I will use the basic format within this video to give me inspiration on how I can do my game better but with a lot more features.
	
	\subsection{Images}
	I found information on how to add images to Javascript \cite{DOMImage}. This gave me a lot of ideas on how I want to implement my images. I realized that I want to use Javascript for my images as it means that I don't need to edit the CSS every time I use an image which can adversely effect the styling on the rest of the page. This can lead to inconsistency issues within the style sheet. I also realised that there are many ways to actually implement images, these being from URLs, video frames, linking it to an image with a script and using them from the same page. The images that I will be implementing will be both static and animated. Further research led me to find a fantastic article on how to implement my ideas \cite{JavascriptImage}. I will make variables which will hold the path to my images and then use the img src tag which will contain the image dimensions as well as the file name. Images are a fantastic way to capture the players attention as it helps create a nice atmosphere, tells a vibrant story without the need to over indulge the player with a heavy text load. I will use these partnered with a descriptor for the area that they are in, the monsters, bakery etc. 
	
	\subsection{Audio} 
	I researched into multiple audio documents. HTML DOM Audio Objects \cite{DOMAudio} contained a lot of the information that I was thinking about for the project. This being automatic audio on button presses, background audio and loopable audio. Within the audio objects documentation, I found the Audio Loop page \cite{DOMAudioLoop} and the Auto Audio page \cite{DOMAudioAuto}. My reason for exploring further into these sections was due to me wanting to create the most immersive experience for the player that I possibly can. The audio will loop after a full cycle and it will play without the user doing anything to start it. This led me to wonder how I was going to go about implementing such a solution when I came across an article by Saruque Ahamed Mollick \cite{AudioAutoArticle}. This has full Javascript on how to play audio only after the page loads using window.onload. document.getElementById(audio).play() will be used to play the audio file. I will use this as inspiration to create my own.
	
	\subsection{Colour Schemes}
	I started by researching into the different colour schemes available which I found within an article by Iveta Pavlova \cite{ColourSchemes}. There were bright, dark and monochrome variants, but ultimately I decided on a softer approach, this being pastel colours. They are fun, joyful and tie in heavily with baked goods. This colour scheme symbolises part of the soft story telling that I want the player to experience. \\
	\figuremacro{h}{pastel}{Pastel Colours}{ - Family of joyful, soothing colours \cite{PastelColours}}{1.0}
  
	\subsection{Cookies}
	Previously I had no idea how to incorporate cookies. I needed to look into the multiple ways to store data within a browser. I analysed this and found sessionStorage, localStorage and javascript cookies. I ultimately decided on the latter \cite{JavascriptCookies}. This increased my understanding of the storage method in general. I am going to create cookies with document.cookies, this will be implemented to be used to store the players in-game name, character creation choices and held items so that it can be held over multiple sessions 
	
	\subsection{Favicon}
	I researched how to incorporate a favicon onto my site \cite{Favicon}. I feel that it adds a piece of my game to the tab of a browser. I will be creating my own favicon for Daedalus Delicatessen through the site favicon.cc found within 101computing.net which I linked above.

    \section{Features}
    There is a large number of different features that I will be incorporating within my game. I will go into detail about my reasoning behind these choices now. 
    
    \subsection{Player Design Choices}
    The player will input a character name of their choice, first and surname. Gender options being male, female, non-binary or alien. One of three starting bakery goods; icing, sponge or a cake casing and finally they can pick a unique colour for their shirt. All of this information will be displayed at the side of the screen. Their will be a character portrait and the players inventory alongside it so that they will know what ingredients and items they have at all times. The baked goods piece will be used to bake a cake in the future as that is one of the main components of the game. The reason I have decided to give the player so many design choices is that I believe it is what creates the most immersion and is a key factor of role-playing. Having control over multiple aspects of your character enthralls the player from the very start and makes them thirsty for more of what the game has to offer.
    
    \subsection{Data Storage}
    Cookies are stored in small text files within your browser. Normally when a web page on the server shuts down, the server forgets everything about the user. We use cookies to remedy this. I will be creating cookies using document.cookie to store all the information stated within the design choices section above, and items found on the players journey.
    
    \subsection{Audio}
    asdf
    
    \subsection{Images}
    asdf
    
    \subsection{Multiple Paths}
    asdf
    
    \section{Navigation Tree}
    '4. A navigation tree and some discussion of how you plan to organise your site. '

    \section{User Interface}
    I chose this colour story over other colour stories because I thought it was the most pleasant to look at, as darker schemes were too somber and neon colours were harsh on the eyes. I feel this colour scheme will work well for my game as it is joyful and enticing to the audience. As mentioned before it ties in with baked goods which will create an immersive atmosphere for the player. Finally the colour story adds a unique feel to my game, many RPG’s exist that are dark and gloomy eg ‘Skyrim’ and ‘Red Dead Redemption’. Whereas my game is light and optimistic.
	
    '5. A sketch of an initial user interface for your system and some commentary on
    the motivation for your design, i.e. how does your design address the features
    you’ve listed. NB. Any designs can be hand-drawn and scanned/photographed for
    inclusion in your report'
    
    \section{Additional Information}
    '6. As appropriate: any additional sections that you deem fit to describe your
    project. For example, if you have decided to implement a particular feature as
    an extension then this would be the place to report on it. Similarly if you intend
    to save data within the browse, then some description of the kinds of data that
    you intend to store, how you will store it, and how it will be structured, should
    be reported on. '

    \section{Totally Optional Appendix}
    If more than 8 pages of text :D

\section{Conclusion}	
\bibliographystyle{ieeetr}
\bibliography{!references}
		
\end{document}
% #######################################
% ########### FILL THESE IN #############
% #######################################
\def\mytitle{Coursework Report}
\def\mykeywords{Deadalus Delicatessen, Monster Bakery, SET08101, Bruce McClelland}
\def\myauthor{Bruce McClelland}
\def\contact{40209907@live.napier.ac.uk}
\def\mymodule{Web Technology (SET08101)}
% #######################################
% #### YOU DON'T NEED TO TOUCH BELOW ####
% #######################################
\documentclass[10pt, a4paper]{article}
\usepackage[a4paper,outer=1.5cm,inner=1.5cm,top=1.75cm,bottom=1.5cm]{geometry}
\twocolumn
\usepackage{graphicx}
\graphicspath{{./images/}}
%colour our links, remove weird boxes
\usepackage[colorlinks,linkcolor={black},citecolor={blue!80!black},urlcolor={blue!80!black}]{hyperref}
%Stop indentation on new paragraphs
\usepackage[parfill]{parskip}
%% Arial-like font
\usepackage{lmodern}
\renewcommand*\familydefault{\sfdefault}
%Napier logo top right
\usepackage{watermark}
%Lorem Ipusm dolor please don't leave any in you final report ;)
\usepackage{lipsum}
\usepackage{xcolor}
\usepackage{listings}
%give us the Capital H that we all know and love
\usepackage{float}
%tone down the line spacing after section titles
\usepackage{titlesec}
%Cool maths printing
\usepackage{amsmath}
%PseudoCode
\usepackage{algorithm2e}

\titlespacing{\subsection}{0pt}{\parskip}{-3pt}
\titlespacing{\subsubsection}{0pt}{\parskip}{-\parskip}
\titlespacing{\paragraph}{0pt}{\parskip}{\parskip}
\newcommand{\figuremacro}[5]{
    \begin{figure}[#1]
        \centering
        \includegraphics[width=#5\columnwidth]{#2}
        \caption[#3]{\textbf{#3}#4}
        \label{fig:#2}
    \end{figure}
}

\lstset{
	escapeinside={/*@}{@*/}, language=C++,
	basicstyle=\fontsize{8.5}{12}\selectfont,
	numbers=left,numbersep=2pt,xleftmargin=2pt,frame=tb,
    columns=fullflexible,showstringspaces=false,tabsize=4,
    keepspaces=true,showtabs=false,showspaces=false,
    backgroundcolor=\color{white}, morekeywords={inline,public,
    class,private,protected,struct},captionpos=t,lineskip=-0.4em,
	aboveskip=10pt, extendedchars=true, breaklines=true,
	prebreak = \raisebox{0ex}[0ex][0ex]{\ensuremath{\hookleftarrow}},
	keywordstyle=\color[rgb]{0,0,1},
	commentstyle=\color[rgb]{0.133,0.545,0.133},
	stringstyle=\color[rgb]{0.627,0.126,0.941}
}

\thiswatermark{\centering \put(336.5,-38.0){\includegraphics[scale=0.8]{logo}} }
\title{\mytitle}
\author{\myauthor\hspace{1em}\\\contact\\Edinburgh Napier University\hspace{0.5em}-\hspace{0.5em}\mymodule}
\date{}
\hypersetup{pdfauthor=\myauthor,pdftitle=\mytitle,pdfkeywords=\mykeywords}
\sloppy
% #######################################
% ########### START FROM HERE ###########
% #######################################
\begin{document}
    \maketitle
    \begin{abstract}
        %Replace the lipsum command with actual text 
        \lipsum[2]
    \end{abstract}
    
    \textbf{Keywords -- }{\mykeywords}
    
    \section{Game Descriptor}
	My game is a free-to-play browser based adventure-RPG where the player controls a character. They select their gender,
	name and starting item which will be with them until used. There will be a myriad of game changing choices. 
	These options have an impact on future dialogue as well as game length. The game can also be reset easily if a mistake
	was made or just for a more wholesome experience if the player wishes to experience all it has to offer.
	
	\subsection{Adventure Based Game}
	\\
	Role Playing Game (RPG) is a genre that creates incredible depth and atmosphere whilst you take control of a character(s)
	within a fictional world. There is a lot of hybrid-genres available within the mothership genre, so it is a very
	challenging genere to describe.\\
	Some sub-genres include; 
	\begin{itemize}
		\item Strategy-RPG
		\item Adventure-RPG
		\item Online-RPG
		\item Action-RPG
	\end{itemize}
	For the purpose of this report, we will talk in greater detail about text-based adventure RPGs.
	Text-based adventure RPG games have been around for a very long time. It is largely story driven
	with events unfolding when players make selections from the multiple on-screen options.
	
	\subsection{Story Draft}
	\paragraph{Daedalus Delicatessen} is a small bakery right in the center of the bustling city Cornelia. One of the only
	human populated nations in planet Terra. Monsteria is a condition that has plagued Terra since the dawn of time where
	humans turn into blood thirsty beasts, some of which call Cornelia home. Of course this caused mass panic and with skin
	several times stronger than tungsten, the royal guards are not remotely effective. \\ 
	\hspace{5mm} Child of Queen Lavender, you are a world renowned baker that makes the most delicious cakes. With the
	ancestral power to slay the stomachs of even the fiercest creatures, you are the worlds only hope to restore humankind
	to it's former glory. To do this, you need them the most irresistable cupcakes and become Cornelias only hope!
    \figuremacro{h}{delicatessen}{ImageTitle}{ - Some Descriptive Text}{1.0}
	
	\section{Background Research}
	\cite{Keshav}
	\cite{sessionStorage}
	\cite{localStorage}
	Some common formatting you may need uses these commands for \textbf{Bold Text}, \textit{Italics}, and \underline{underlined}.
	\subsection{LineBreaks}
	Here is a line
    
    Here is a line followed by a double line break.
	This line is only one line break down from the above, Notice that latex can ignore this
    
    We can force a break \\ with the break operator.
    
	\subsection{Maths}
    Embedding Maths is Latex's bread and butter    
    
    {\centering \Large \(
        J = \begin{bmatrix}
            \frac{\delta e}{\delta \theta _0}
            \frac{\delta e}{\delta \theta _1}
            \frac{\delta e}{\delta \theta _2}
        \end{bmatrix}
        = e_{current} - e_{target} 
    \)\par}
	
	\subsection{Code Listing}
    You can load segments of code from a file, or embed them directly.
    
\begin{lstlisting}[caption = Hello World! in c++]
#include <iostream>

int main() {
    std::cout << "Hello World!" << std::endl;
    std::cin.get();
    return 0;
}
\end{lstlisting}

\subsection{PseudoCode}

\begin{algorithm}[h]
\For{$i = 0$ \KwTo $100$}{
 print\_number = true\;
\If{i is divisible by 3}{
 print "Fizz"\;
 print\_number = false\;
}
\If{i is divisible by 5}{
 print "Buzz"\;
 print\_number = false\;
}
\If{print\_number}{
    print i\;
}
print a newline\;
}
\caption{FizzBuzz}
\end{algorithm}
	
\section{Conclusion}	
\bibliographystyle{ieeetr}
\bibliography{references}
		
\end{document}